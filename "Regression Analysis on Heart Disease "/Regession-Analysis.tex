% Options for packages loaded elsewhere
\PassOptionsToPackage{unicode}{hyperref}
\PassOptionsToPackage{hyphens}{url}
%
\documentclass[
]{article}
\usepackage{amsmath,amssymb}
\usepackage{lmodern}
\usepackage{iftex}
\ifPDFTeX
  \usepackage[T1]{fontenc}
  \usepackage[utf8]{inputenc}
  \usepackage{textcomp} % provide euro and other symbols
\else % if luatex or xetex
  \usepackage{unicode-math}
  \defaultfontfeatures{Scale=MatchLowercase}
  \defaultfontfeatures[\rmfamily]{Ligatures=TeX,Scale=1}
\fi
% Use upquote if available, for straight quotes in verbatim environments
\IfFileExists{upquote.sty}{\usepackage{upquote}}{}
\IfFileExists{microtype.sty}{% use microtype if available
  \usepackage[]{microtype}
  \UseMicrotypeSet[protrusion]{basicmath} % disable protrusion for tt fonts
}{}
\makeatletter
\@ifundefined{KOMAClassName}{% if non-KOMA class
  \IfFileExists{parskip.sty}{%
    \usepackage{parskip}
  }{% else
    \setlength{\parindent}{0pt}
    \setlength{\parskip}{6pt plus 2pt minus 1pt}}
}{% if KOMA class
  \KOMAoptions{parskip=half}}
\makeatother
\usepackage{xcolor}
\usepackage[margin=1in]{geometry}
\usepackage{color}
\usepackage{fancyvrb}
\newcommand{\VerbBar}{|}
\newcommand{\VERB}{\Verb[commandchars=\\\{\}]}
\DefineVerbatimEnvironment{Highlighting}{Verbatim}{commandchars=\\\{\}}
% Add ',fontsize=\small' for more characters per line
\usepackage{framed}
\definecolor{shadecolor}{RGB}{248,248,248}
\newenvironment{Shaded}{\begin{snugshade}}{\end{snugshade}}
\newcommand{\AlertTok}[1]{\textcolor[rgb]{0.94,0.16,0.16}{#1}}
\newcommand{\AnnotationTok}[1]{\textcolor[rgb]{0.56,0.35,0.01}{\textbf{\textit{#1}}}}
\newcommand{\AttributeTok}[1]{\textcolor[rgb]{0.77,0.63,0.00}{#1}}
\newcommand{\BaseNTok}[1]{\textcolor[rgb]{0.00,0.00,0.81}{#1}}
\newcommand{\BuiltInTok}[1]{#1}
\newcommand{\CharTok}[1]{\textcolor[rgb]{0.31,0.60,0.02}{#1}}
\newcommand{\CommentTok}[1]{\textcolor[rgb]{0.56,0.35,0.01}{\textit{#1}}}
\newcommand{\CommentVarTok}[1]{\textcolor[rgb]{0.56,0.35,0.01}{\textbf{\textit{#1}}}}
\newcommand{\ConstantTok}[1]{\textcolor[rgb]{0.00,0.00,0.00}{#1}}
\newcommand{\ControlFlowTok}[1]{\textcolor[rgb]{0.13,0.29,0.53}{\textbf{#1}}}
\newcommand{\DataTypeTok}[1]{\textcolor[rgb]{0.13,0.29,0.53}{#1}}
\newcommand{\DecValTok}[1]{\textcolor[rgb]{0.00,0.00,0.81}{#1}}
\newcommand{\DocumentationTok}[1]{\textcolor[rgb]{0.56,0.35,0.01}{\textbf{\textit{#1}}}}
\newcommand{\ErrorTok}[1]{\textcolor[rgb]{0.64,0.00,0.00}{\textbf{#1}}}
\newcommand{\ExtensionTok}[1]{#1}
\newcommand{\FloatTok}[1]{\textcolor[rgb]{0.00,0.00,0.81}{#1}}
\newcommand{\FunctionTok}[1]{\textcolor[rgb]{0.00,0.00,0.00}{#1}}
\newcommand{\ImportTok}[1]{#1}
\newcommand{\InformationTok}[1]{\textcolor[rgb]{0.56,0.35,0.01}{\textbf{\textit{#1}}}}
\newcommand{\KeywordTok}[1]{\textcolor[rgb]{0.13,0.29,0.53}{\textbf{#1}}}
\newcommand{\NormalTok}[1]{#1}
\newcommand{\OperatorTok}[1]{\textcolor[rgb]{0.81,0.36,0.00}{\textbf{#1}}}
\newcommand{\OtherTok}[1]{\textcolor[rgb]{0.56,0.35,0.01}{#1}}
\newcommand{\PreprocessorTok}[1]{\textcolor[rgb]{0.56,0.35,0.01}{\textit{#1}}}
\newcommand{\RegionMarkerTok}[1]{#1}
\newcommand{\SpecialCharTok}[1]{\textcolor[rgb]{0.00,0.00,0.00}{#1}}
\newcommand{\SpecialStringTok}[1]{\textcolor[rgb]{0.31,0.60,0.02}{#1}}
\newcommand{\StringTok}[1]{\textcolor[rgb]{0.31,0.60,0.02}{#1}}
\newcommand{\VariableTok}[1]{\textcolor[rgb]{0.00,0.00,0.00}{#1}}
\newcommand{\VerbatimStringTok}[1]{\textcolor[rgb]{0.31,0.60,0.02}{#1}}
\newcommand{\WarningTok}[1]{\textcolor[rgb]{0.56,0.35,0.01}{\textbf{\textit{#1}}}}
\usepackage{graphicx}
\makeatletter
\def\maxwidth{\ifdim\Gin@nat@width>\linewidth\linewidth\else\Gin@nat@width\fi}
\def\maxheight{\ifdim\Gin@nat@height>\textheight\textheight\else\Gin@nat@height\fi}
\makeatother
% Scale images if necessary, so that they will not overflow the page
% margins by default, and it is still possible to overwrite the defaults
% using explicit options in \includegraphics[width, height, ...]{}
\setkeys{Gin}{width=\maxwidth,height=\maxheight,keepaspectratio}
% Set default figure placement to htbp
\makeatletter
\def\fps@figure{htbp}
\makeatother
\setlength{\emergencystretch}{3em} % prevent overfull lines
\providecommand{\tightlist}{%
  \setlength{\itemsep}{0pt}\setlength{\parskip}{0pt}}
\setcounter{secnumdepth}{-\maxdimen} % remove section numbering
\ifLuaTeX
  \usepackage{selnolig}  % disable illegal ligatures
\fi
\IfFileExists{bookmark.sty}{\usepackage{bookmark}}{\usepackage{hyperref}}
\IfFileExists{xurl.sty}{\usepackage{xurl}}{} % add URL line breaks if available
\urlstyle{same} % disable monospaced font for URLs
\hypersetup{
  pdftitle={Regression Analysis on Heart Disease},
  pdfauthor={Anjali Priya},
  hidelinks,
  pdfcreator={LaTeX via pandoc}}

\title{Regression Analysis on Heart Disease}
\author{Anjali Priya}
\date{2022-09-14}

\begin{document}
\maketitle

\hypertarget{r-markdown}{%
\subsection{R Markdown}\label{r-markdown}}

This is an R Markdown document. Markdown is a simple formatting syntax
for authoring HTML, PDF, and MS Word documents. For more details on
using R Markdown see \url{http://rmarkdown.rstudio.com}.

When you click the \textbf{Knit} button a document will be generated
that includes both content as well as the output of any embedded R code
chunks within the document. You can embed an R code chunk like this:

\begin{Shaded}
\begin{Highlighting}[]
\FunctionTok{setwd}\NormalTok{(}\StringTok{"C:/personal files/data analytics/docs/MODULE 3/T3"}\NormalTok{)}
\FunctionTok{getwd}\NormalTok{()}
\end{Highlighting}
\end{Shaded}

\begin{verbatim}
## [1] "C:/personal files/data analytics/docs/MODULE 3/T3"
\end{verbatim}

loading library

\begin{Shaded}
\begin{Highlighting}[]
\FunctionTok{library}\NormalTok{(ggplot2)}
\FunctionTok{library}\NormalTok{(tidyr)}
\FunctionTok{library}\NormalTok{(readxl)}
\FunctionTok{library}\NormalTok{(readr)}
\FunctionTok{library}\NormalTok{(car)}
\end{Highlighting}
\end{Shaded}

\begin{verbatim}
## Loading required package: carData
\end{verbatim}

\begin{Shaded}
\begin{Highlighting}[]
\FunctionTok{library}\NormalTok{(tidyverse)}
\end{Highlighting}
\end{Shaded}

\begin{verbatim}
## -- Attaching packages --------------------------------------- tidyverse 1.3.1 --
\end{verbatim}

\begin{verbatim}
## v tibble  3.1.6      v stringr 1.4.0 
## v purrr   0.3.4      v forcats 0.5.1 
## v dplyr   1.0.10
\end{verbatim}

\begin{verbatim}
## -- Conflicts ------------------------------------------ tidyverse_conflicts() --
## x dplyr::filter() masks stats::filter()
## x dplyr::lag()    masks stats::lag()
## x dplyr::recode() masks car::recode()
## x purrr::some()   masks car::some()
\end{verbatim}

\begin{Shaded}
\begin{Highlighting}[]
\FunctionTok{library}\NormalTok{(corrplot)}
\end{Highlighting}
\end{Shaded}

\begin{verbatim}
## corrplot 0.92 loaded
\end{verbatim}

\begin{Shaded}
\begin{Highlighting}[]
\FunctionTok{library}\NormalTok{(broom)}
\end{Highlighting}
\end{Shaded}

\hypertarget{brief}{%
\subsection{BRIEF}\label{brief}}

The dataset ``heart.data'' contains observations on the percentage of
people cycling to work each day, the percentage of people smoking, and
the percentage of people with heart diseases in a hypothetical sample of
498 towns. The rates of cycling to work range between 1 and 75\%, rates
of smoking between 0.5 and 30\%, and rates of heart disease between
0.5\% and 20.5\%. The Surveyor want to check the relationship between
cycling to work and heart diseases using above data

\begin{quote}
Q1. Read, call and view the data in r
\end{quote}

\begin{Shaded}
\begin{Highlighting}[]
\NormalTok{cardiodata\_1 }\OtherTok{=} \FunctionTok{read\_excel}\NormalTok{(}\StringTok{"C:/personal files/data analytics/docs/MODULE 3/T3/heart.data.xlsx"}\NormalTok{)}
\FunctionTok{View}\NormalTok{(cardiodata\_1)}
\end{Highlighting}
\end{Shaded}

\begin{quote}
Q2. check the dimension and characteristics of the dataset.
\end{quote}

\begin{Shaded}
\begin{Highlighting}[]
\FunctionTok{dim}\NormalTok{(cardiodata\_1)}
\end{Highlighting}
\end{Shaded}

\begin{verbatim}
## [1] 498   3
\end{verbatim}

\begin{Shaded}
\begin{Highlighting}[]
\FunctionTok{class}\NormalTok{(cardiodata\_1)}
\end{Highlighting}
\end{Shaded}

\begin{verbatim}
## [1] "tbl_df"     "tbl"        "data.frame"
\end{verbatim}

\begin{itemize}
\tightlist
\item
  \emph{As seen from the given dataset are in tibble, dataframe class .}
\item
  \emph{Dimension of data set is {[}498 x 3{]} .}
\end{itemize}

\begin{quote}
Q3. check properties of the variables.
\end{quote}

\begin{Shaded}
\begin{Highlighting}[]
\FunctionTok{str}\NormalTok{(cardiodata\_1)}
\end{Highlighting}
\end{Shaded}

\begin{verbatim}
## tibble [498 x 3] (S3: tbl_df/tbl/data.frame)
##  $ cycling       : num [1:498] 30.8 65.13 1.96 44.8 69.43 ...
##  $ smoking       : num [1:498] 10.9 2.22 17.59 2.8 15.97 ...
##  $ heart_diseases: num [1:498] 11.77 2.85 17.18 6.82 4.06 ...
\end{verbatim}

\begin{itemize}
\tightlist
\item
  \emph{As seen from the given data all the variables are in numeric
  form.}
\end{itemize}

\begin{quote}
Q4. Check the first and last few observations from the dataset
\end{quote}

\begin{Shaded}
\begin{Highlighting}[]
\FunctionTok{head}\NormalTok{(cardiodata\_1 , }\DecValTok{10}\NormalTok{)}
\end{Highlighting}
\end{Shaded}

\begin{verbatim}
## # A tibble: 10 x 3
##    cycling smoking heart_diseases
##      <dbl>   <dbl>          <dbl>
##  1   30.8    10.9           11.8 
##  2   65.1     2.22           2.85
##  3    1.96   17.6           17.2 
##  4   44.8     2.80           6.82
##  5   69.4    16.0            4.06
##  6   54.4    29.3            9.55
##  7   49.1     9.06           7.62
##  8    4.78   12.8           15.9 
##  9   65.7    12.0            3.07
## 10   35.3    23.3           12.1
\end{verbatim}

\begin{Shaded}
\begin{Highlighting}[]
\FunctionTok{tail}\NormalTok{(cardiodata\_1, }\DecValTok{10}\NormalTok{)}
\end{Highlighting}
\end{Shaded}

\begin{verbatim}
## # A tibble: 10 x 3
##    cycling smoking heart_diseases
##      <dbl>   <dbl>          <dbl>
##  1   41.6    0.533           5.93
##  2   70.2   16.3             4.71
##  3   50.8    6.03            5.59
##  4   68.9   10.5             3.11
##  5   21.6    7.60           12.4 
##  6   47.7   27.6            11.3 
##  7   45.1   21.4             9.62
##  8    8.28   6.42           13.5 
##  9   42.3   20.7            10.1 
## 10   30.8   23.6            11.8
\end{verbatim}

\begin{quote}
Q5. Create the summary
\end{quote}

\begin{Shaded}
\begin{Highlighting}[]
\FunctionTok{summary}\NormalTok{(cardiodata\_1)}
\end{Highlighting}
\end{Shaded}

\begin{verbatim}
##     cycling          smoking        heart_diseases   
##  Min.   : 1.119   Min.   : 0.5259   Min.   : 0.5519  
##  1st Qu.:20.205   1st Qu.: 8.2798   1st Qu.: 6.5137  
##  Median :35.824   Median :15.8146   Median :10.3853  
##  Mean   :37.788   Mean   :15.4350   Mean   :10.1745  
##  3rd Qu.:57.853   3rd Qu.:22.5689   3rd Qu.:13.7240  
##  Max.   :74.907   Max.   :29.9467   Max.   :20.4535
\end{verbatim}

\begin{Shaded}
\begin{Highlighting}[]
\NormalTok{cardiodata\_1 }\SpecialCharTok{\%\textgreater{}\%}
  \FunctionTok{boxplot}\NormalTok{( }\AttributeTok{main =} \StringTok{"Boxplot of percentage distribution "}\NormalTok{ , }\AttributeTok{ylab=}\StringTok{"percent"}\NormalTok{)}
\end{Highlighting}
\end{Shaded}

\includegraphics{Regession-Analysis_files/figure-latex/unnamed-chunk-7-1.pdf}

\begin{Shaded}
\begin{Highlighting}[]
\FunctionTok{lapply}\NormalTok{(cardiodata\_1, boxplot.stats)}
\end{Highlighting}
\end{Shaded}

\begin{verbatim}
## $cycling
## $cycling$stats
## [1]  1.119154 20.197206 35.824459 57.978406 74.907111
## 
## $cycling$n
## [1] 498
## 
## $cycling$conf
## [1] 33.14949 38.49942
## 
## $cycling$out
## numeric(0)
## 
## 
## $smoking
## $smoking$stats
## [1]  0.525850  8.278009 15.814614 22.585020 29.946743
## 
## $smoking$n
## [1] 498
## 
## $smoking$conf
## [1] 14.80166 16.82757
## 
## $smoking$out
## numeric(0)
## 
## 
## $heart_diseases
## $heart_diseases$stats
## [1]  0.5518982  6.5126756 10.3852547 13.7241183 20.4534962
## 
## $heart_diseases$n
## [1] 498
## 
## $heart_diseases$conf
## [1]  9.874674 10.895836
## 
## $heart_diseases$out
## numeric(0)
\end{verbatim}

\begin{Shaded}
\begin{Highlighting}[]
\DocumentationTok{\#\# standarizing data }
\NormalTok{cardiodata}\OtherTok{=}\FunctionTok{as.data.frame}\NormalTok{( }\FunctionTok{scale}\NormalTok{(cardiodata\_1))}
\NormalTok{cardiodata }\SpecialCharTok{\%\textgreater{}\%}
  \FunctionTok{boxplot}\NormalTok{( }\AttributeTok{main =} \StringTok{"Boxplot of percentage distribution "}\NormalTok{ , }\AttributeTok{ylab=}\StringTok{"percent"}\NormalTok{)}
\end{Highlighting}
\end{Shaded}

\includegraphics{Regession-Analysis_files/figure-latex/unnamed-chunk-7-2.pdf}

\begin{itemize}
\tightlist
\item
  \emph{data under each category is normally distributed around median}
\item
  \emph{data under cycling category has largest spread and higher median
  for 498 towns}
\end{itemize}

\begin{quote}
Q6. Check normality of dependent variable and linearity between
variables
\end{quote}

\begin{Shaded}
\begin{Highlighting}[]
\FunctionTok{qqPlot}\NormalTok{(cardiodata}\SpecialCharTok{$}\NormalTok{heart\_diseases)}
\end{Highlighting}
\end{Shaded}

\includegraphics{Regession-Analysis_files/figure-latex/unnamed-chunk-8-1.pdf}

\begin{verbatim}
## [1] 215 284
\end{verbatim}

\begin{Shaded}
\begin{Highlighting}[]
\NormalTok{correlationmatrix }\OtherTok{=} \FunctionTok{cor}\NormalTok{(cardiodata)}
\FunctionTok{corrplot}\NormalTok{(}\FunctionTok{cor}\NormalTok{(correlationmatrix),}\AttributeTok{method =}\StringTok{"number"}\NormalTok{)}
\end{Highlighting}
\end{Shaded}

\includegraphics{Regession-Analysis_files/figure-latex/unnamed-chunk-8-2.pdf}

\begin{Shaded}
\begin{Highlighting}[]
\FunctionTok{plot}\NormalTok{(cardiodata)}
\end{Highlighting}
\end{Shaded}

\includegraphics{Regession-Analysis_files/figure-latex/unnamed-chunk-8-3.pdf}

\begin{itemize}
\tightlist
\item
  Seeing the qqplot we can say that the data are approximately normally
  distributed
\item
  There is high correlation negative correlation between cycling-heart
  disease , weak positive correlation between smoking and heart disease
  and no correlation between cycling and smoking*
\item
  *No outliers in dataset
\end{itemize}

\begin{quote}
Q7. To check if there is a linear relationship between ``cycling to
work'', ``smoking'', and ``heart disease'' in our hypothetical survey of
498 towns. Create and run a regression model using ``heart.data''
dataset and also create summary of the regression model
\end{quote}

\begin{Shaded}
\begin{Highlighting}[]
\NormalTok{model\_1}\OtherTok{=}\FunctionTok{lm}\NormalTok{(heart\_diseases}\SpecialCharTok{\textasciitilde{}}\NormalTok{ cycling }\SpecialCharTok{+}\NormalTok{ smoking ,cardiodata )}
\FunctionTok{summary}\NormalTok{(model\_1)}
\end{Highlighting}
\end{Shaded}

\begin{verbatim}
## 
## Call:
## lm(formula = heart_diseases ~ cycling + smoking, data = cardiodata)
## 
## Residuals:
##      Min       1Q   Median       3Q      Max 
## -0.47658 -0.09762  0.00792  0.09671  0.42283 
## 
## Coefficients:
##               Estimate Std. Error t value Pr(>|t|)    
## (Intercept) -6.119e-17  6.410e-03    0.00        1    
## cycling     -9.403e-01  6.418e-03 -146.53   <2e-16 ***
## smoking      3.234e-01  6.418e-03   50.39   <2e-16 ***
## ---
## Signif. codes:  0 '***' 0.001 '**' 0.01 '*' 0.05 '.' 0.1 ' ' 1
## 
## Residual standard error: 0.1431 on 495 degrees of freedom
## Multiple R-squared:  0.9796, Adjusted R-squared:  0.9795 
## F-statistic: 1.19e+04 on 2 and 495 DF,  p-value: < 2.2e-16
\end{verbatim}

\begin{Shaded}
\begin{Highlighting}[]
\NormalTok{model\_2}\OtherTok{=}\FunctionTok{lm}\NormalTok{(heart\_diseases }\SpecialCharTok{\textasciitilde{}}\NormalTok{ cycling , cardiodata)}
\FunctionTok{summary}\NormalTok{(model\_2)}
\end{Highlighting}
\end{Shaded}

\begin{verbatim}
## 
## Call:
## lm(formula = heart_diseases ~ cycling, data = cardiodata)
## 
## Residuals:
##      Min       1Q   Median       3Q      Max 
## -0.88111 -0.26369 -0.00088  0.25179  0.79674 
## 
## Coefficients:
##               Estimate Std. Error t value Pr(>|t|)    
## (Intercept) -5.015e-17  1.585e-02    0.00        1    
## cycling     -9.355e-01  1.587e-02  -58.94   <2e-16 ***
## ---
## Signif. codes:  0 '***' 0.001 '**' 0.01 '*' 0.05 '.' 0.1 ' ' 1
## 
## Residual standard error: 0.3538 on 496 degrees of freedom
## Multiple R-squared:  0.8751, Adjusted R-squared:  0.8748 
## F-statistic:  3474 on 1 and 496 DF,  p-value: < 2.2e-16
\end{verbatim}

\begin{itemize}
\tightlist
\item
  As seen from the summary stats of linear regression its evident that
  including cycling and smoking both give the better fit model by
  looking at the Adjusted R sqrd value which is 97.6\% and all the
  parameters are highly significant.
\end{itemize}

\begin{quote}
Q8. Store the output of the regression model and print coefficients.
\end{quote}

\begin{Shaded}
\begin{Highlighting}[]
\NormalTok{model\_1}
\end{Highlighting}
\end{Shaded}

\begin{verbatim}
## 
## Call:
## lm(formula = heart_diseases ~ cycling + smoking, data = cardiodata)
## 
## Coefficients:
## (Intercept)      cycling      smoking  
##  -6.119e-17   -9.403e-01    3.234e-01
\end{verbatim}

\begin{Shaded}
\begin{Highlighting}[]
\NormalTok{coff\_m1}\OtherTok{=}\NormalTok{model\_1}\SpecialCharTok{$}\NormalTok{coefficients}
\end{Highlighting}
\end{Shaded}

\begin{quote}
Q9. Print the Estimate, Std. Error, t-value, and p-value for the
independent variables (i.e., cycling and smoking)
\end{quote}

\begin{Shaded}
\begin{Highlighting}[]
\FunctionTok{summary}\NormalTok{(model\_1)}\SpecialCharTok{$}\NormalTok{coefficients }
\end{Highlighting}
\end{Shaded}

\begin{verbatim}
##                  Estimate  Std. Error       t value      Pr(>|t|)
## (Intercept) -6.119484e-17 0.006410474 -9.546071e-15  1.000000e+00
## cycling     -9.403500e-01 0.006417655 -1.465255e+02  0.000000e+00
## smoking      3.233643e-01 0.006417655  5.038667e+01 5.192235e-197
\end{verbatim}

\begin{quote}
Q10. Print residual standard error, r-squared, adjusted r-squared,
f-statistic, p-value from the output
\end{quote}

\begin{Shaded}
\begin{Highlighting}[]
\NormalTok{RSE }\OtherTok{=} \FunctionTok{summary}\NormalTok{(model\_1)}\SpecialCharTok{$}\NormalTok{sigma}
\NormalTok{R\_sqrd}\OtherTok{=}\FunctionTok{summary}\NormalTok{(model\_1)}\SpecialCharTok{$}\NormalTok{r.squared}
\NormalTok{AdjR\_sqrd}\OtherTok{=}\FunctionTok{summary}\NormalTok{(model\_1)}\SpecialCharTok{$}\NormalTok{adj.r.squared}
\NormalTok{F\_value}\OtherTok{=}\FunctionTok{summary}\NormalTok{(model\_1)}\SpecialCharTok{$}\NormalTok{fstatistic}
\end{Highlighting}
\end{Shaded}

\begin{quote}
Q11. Compute the confidence intervals
\end{quote}

\begin{Shaded}
\begin{Highlighting}[]
\NormalTok{conf\_interval}\OtherTok{=}\FunctionTok{predict}\NormalTok{(model\_1 , cardiodata , }\AttributeTok{interval =}\StringTok{"confidence"}\NormalTok{)}
\FunctionTok{head}\NormalTok{(conf\_interval,}\DecValTok{10}\NormalTok{)}
\end{Highlighting}
\end{Shaded}

\begin{verbatim}
##           fit        lwr        upr
## 1   0.1288328  0.1139247  0.1437409
## 2  -1.7123317 -1.7411415 -1.6835219
## 3   1.6523906  1.6276168  1.6771644
## 4  -0.7996914 -0.8230828 -0.7763000
## 5  -1.3639947 -1.3864395 -1.3415499
## 6  -0.1852071 -0.2115583 -0.1588559
## 7  -0.7418803 -0.7591522 -0.7246083
## 8   1.3433182  1.3199248  1.3667116
## 9  -1.3575020 -1.3788956 -1.3361084
## 10  0.4167086  0.3992823  0.4341350
\end{verbatim}

\begin{quote}
Q12. Create the diagnostic plots and discuss them in brief.
\end{quote}

\begin{Shaded}
\begin{Highlighting}[]
\FunctionTok{plot}\NormalTok{(model\_1)}
\end{Highlighting}
\end{Shaded}

\includegraphics{Regession-Analysis_files/figure-latex/unnamed-chunk-14-1.pdf}
\includegraphics{Regession-Analysis_files/figure-latex/unnamed-chunk-14-2.pdf}
\includegraphics{Regession-Analysis_files/figure-latex/unnamed-chunk-14-3.pdf}
\includegraphics{Regession-Analysis_files/figure-latex/unnamed-chunk-14-4.pdf}

\begin{Shaded}
\begin{Highlighting}[]
\NormalTok{heart\_fitted}\OtherTok{=}\FunctionTok{augment}\NormalTok{(model\_1)}\SpecialCharTok{$}\NormalTok{.fitted}
\end{Highlighting}
\end{Shaded}

\begin{itemize}
\item
  Residual vs Fitted value has uniform spread and the red lines is
  deviates a little at ends but overall its more or less is perfect
  straight across plot .Hence we can declare that the residual follow
  linear pattern.
\item
  We can see that in QQplot the points fall roughly along the straight
  line , Hence points are normally distributed
\item
  Residual vs Fitted value has uniform spread and the red lines is
  almost straight across plot .Hence we can declare that the assumption
  of equal variance in not violated hence the model ids homoscedastic.
\item
  In Residual vs Leverage we can see that observation 196 i s on cooks
  line but doesnot cross it .This means there is no potential
  influencers.
\end{itemize}

\begin{quote}
Q13. Check autocorrelation and heteroscedasticity using appropriate
statistical test.
\end{quote}

\begin{itemize}
\tightlist
\item
  \textbf{Test for Autocorrelation with the ACF Plot}
\end{itemize}

\begin{Shaded}
\begin{Highlighting}[]
\NormalTok{residuals\_mod1 }\OtherTok{=}\NormalTok{model\_1}\SpecialCharTok{$}\NormalTok{residuals}
\FunctionTok{acf}\NormalTok{(residuals\_mod1 , }\AttributeTok{type =}\StringTok{"correlation"}\NormalTok{)}
\end{Highlighting}
\end{Shaded}

\includegraphics{Regession-Analysis_files/figure-latex/unnamed-chunk-15-1.pdf}

\emph{After the lag-0 correlation, the subsequent correlations drop
quickly to zero and stay (mostly) between the limits of the significance
level (dashed blue lines). Therefore, we can conclude that the residuals
of this model meet the assumption of no autocorrelation.}

\begin{itemize}
\tightlist
\item
  \textbf{Durbin-Watson Test to Check Autocorrelation}

  \begin{itemize}
  \tightlist
  \item
    H0 : First order autocorrelation do not exist
  \item
    H1 : First order autocorrelation exist
  \end{itemize}
\end{itemize}

\begin{Shaded}
\begin{Highlighting}[]
\CommentTok{\#Durbin{-}Watson Test to Check Autocorrelation}

\FunctionTok{library}\NormalTok{(lmtest)}
\end{Highlighting}
\end{Shaded}

\begin{verbatim}
## Loading required package: zoo
\end{verbatim}

\begin{verbatim}
## 
## Attaching package: 'zoo'
\end{verbatim}

\begin{verbatim}
## The following objects are masked from 'package:base':
## 
##     as.Date, as.Date.numeric
\end{verbatim}

\begin{Shaded}
\begin{Highlighting}[]
\NormalTok{lmtest}\SpecialCharTok{::}\FunctionTok{dwtest}\NormalTok{(model\_1)}
\end{Highlighting}
\end{Shaded}

\begin{verbatim}
## 
##  Durbin-Watson test
## 
## data:  model_1
## DW = 1.9174, p-value = 0.1773
## alternative hypothesis: true autocorrelation is greater than 0
\end{verbatim}

\begin{itemize}
\tightlist
\item
  \emph{As the DW value is 1.91 which lies in the range of 1.5 to 2.5
  hence there is no autocorrelation between the residuals}
\end{itemize}

\textbf{Testing Heteroskedacity}

\begin{itemize}
\tightlist
\item
  \textbf{Perform the Breusch--Pagan Test to Check Heteroscedasticity}

  \begin{itemize}
  \tightlist
  \item
    H0 : Residuals are distributed with equal variance(i.e
    homoscedastic)
  \item
    H1 : Residuals are distributes with non equal variance(i.e
    heteroscedastic)
  \end{itemize}
\end{itemize}

\begin{Shaded}
\begin{Highlighting}[]
\NormalTok{lmtest}\SpecialCharTok{::}\FunctionTok{bgtest}\NormalTok{(model\_1)}
\end{Highlighting}
\end{Shaded}

\begin{verbatim}
## 
##  Breusch-Godfrey test for serial correlation of order up to 1
## 
## data:  model_1
## LM test = 0.63776, df = 1, p-value = 0.4245
\end{verbatim}

\begin{itemize}
\tightlist
\item
  \emph{p value is greater than .05 hence we fail to reject the null
  confirming that model is homoscedastic}
\end{itemize}

\begin{Shaded}
\begin{Highlighting}[]
\FunctionTok{plot}\NormalTok{(cardiodata}\SpecialCharTok{$}\NormalTok{heart\_diseases ,heart\_fitted , }\AttributeTok{xlab=} \StringTok{"Predicted Heart disease"}\NormalTok{ , }\AttributeTok{ylab=}\StringTok{"Actual Heart disease"}\NormalTok{)}
\end{Highlighting}
\end{Shaded}

\includegraphics{Regession-Analysis_files/figure-latex/unnamed-chunk-18-1.pdf}

\begin{Shaded}
\begin{Highlighting}[]
\DocumentationTok{\#\# unscaling data for accuracy check and rmse of model}
\NormalTok{targetmean }\OtherTok{=} \FunctionTok{mean}\NormalTok{(cardiodata\_1}\SpecialCharTok{$}\NormalTok{heart\_diseases)}
\NormalTok{targetsd }\OtherTok{=} \FunctionTok{sd}\NormalTok{(cardiodata\_1}\SpecialCharTok{$}\NormalTok{heart\_diseases)}
\NormalTok{unscaledtest.obs }\OtherTok{=} \FunctionTok{round}\NormalTok{(cardiodata}\SpecialCharTok{$}\NormalTok{heart\_diseases }\SpecialCharTok{*}\NormalTok{targetsd }\SpecialCharTok{+}\NormalTok{ targetmean , }\DecValTok{0}\NormalTok{)}
\NormalTok{unscaledtest.pred }\OtherTok{=} \FunctionTok{round}\NormalTok{ (heart\_fitted }\SpecialCharTok{*}\NormalTok{targetsd }\SpecialCharTok{+}\NormalTok{ targetmean , }\DecValTok{0}\NormalTok{)}
\DocumentationTok{\#\#\#\#\#}


\NormalTok{Accuracy}\OtherTok{=}\FunctionTok{mean}\NormalTok{(unscaledtest.obs }\SpecialCharTok{==}\NormalTok{unscaledtest.pred )}
\NormalTok{Accuracy}
\end{Highlighting}
\end{Shaded}

\begin{verbatim}
## [1] 0.4959839
\end{verbatim}

\begin{Shaded}
\begin{Highlighting}[]
\NormalTok{rmse}\OtherTok{=} \FunctionTok{sqrt}\NormalTok{(}\FunctionTok{mean}\NormalTok{((unscaledtest.obs}\SpecialCharTok{{-}}\NormalTok{unscaledtest.pred)}\SpecialCharTok{\^{}}\DecValTok{2}\NormalTok{))}
\NormalTok{rmse}
\end{Highlighting}
\end{Shaded}

\begin{verbatim}
## [1] 0.8127992
\end{verbatim}

\end{document}
